\section{Need of Project} \label{sec:need}

A robot is a container for AI, sometimes mimicking the human form, sometimes not�but the AI itself is the computer inside the robot. AI is the brain, and the robot is its body�if it even has a body. For example, the software and data behind Siri is AI, the woman�s voice we hear is a personification of that AI, and there�s no robot involved at all.

The term "`Singularity"'  has been used in math to describe an asymptote-like situation where normal rules no longer apply. It�s been used in physics to describe a phenomenon like an infinitely small, dense black hole or the point we were all squished into right before the Big Bang. Again, situations where the usual rules don�t apply. 

There are many different types or forms of AI since AI is a broad concept, the critical categories we need to think about are based on an AI�s caliber. There are three major AI caliber categories:

\begin{itemize}
	\item Artificial Narrow Intelligence (ANI): Sometimes referred to as Weak AI, Artificial Narrow Intelligence is AI that specializes in one area. There�s AI that can beat the world chess champion in chess, but that�s the only thing it does. Ask it to figure out a better way to store data on a hard drive, and it�ll look at you blankly.
	\item Artificial General Intelligence (AGI): Sometimes referred to as Strong AI, or Human-Level AI, Artificial General Intelligence refers to a computer that is as smart as a human across the board�a machine that can perform any intellectual task that a human being can. Creating AGI is a much harder task than creating ANI, and we�re yet to do it. Professor Linda Gottfredson describes intelligence as �a very general mental capability that, among other things, involves the ability to reason, plan, solve problems, think abstractly, comprehend complex ideas, learn quickly, and learn from experience.� AGI would be able to do all of those things as easily as you can.
	\item Artificial Superintelligence (ASI): Oxford philosopher and leading AI thinker Nick Bostrom defines superintelligence as �an intellect that is much smarter than the best human brains in practically every field, including scientific creativity, general wisdom and social skills.� Artificial Superintelligence ranges from a computer that�s just a little smarter than a human to one that�s trillions of times smarter�across the board. ASI is the reason the topic of AI is such a spicy meatball and why the words �immortality� and �extinction� will both appear in these posts multiple times.
\end{itemize}

As of now, humans have conquered the lowest caliber of AI-ANI-in many ways, and it�s everywhere. The AI Revolution is the road from ANI, through AGI, to ASI�a road we may or may not survive but that, either way, will change everything.
\vspace{5mm}
Nothing will make you appreciate human intelligence like learning about how unbelievably challenging it is to try to create a computer as smart as we are. Building skyscrapers, putting humans in space, figuring out the details of how the Big Bang went down�all far easier than understanding our own brain or how to make something as cool as it. As of now, the human brain is the most complex object in the known universe. What�s interesting is that the hard parts of trying to build AGI are not intuitively what you�d think they are. 

What you quickly realize when you think about this is that those things that seem easy to us are actually unbelievably complicated, and they only seem easy because those skills have been optimized in us (and most animals) by hundreds of millions of years of animal evolution. When you reach your hand up toward an object, the muscles, tendons, and bones in your shoulder, elbow, and wrist instantly perform a long series of physics operations, in conjunction with your eyes, to allow you to move your hand in a straight line through three dimensions. It seems effortless to you because you have perfected software in your brain for doing it. Same idea goes for why it�s not that malware is dumb for not being able to figure out the slanty word recognition test when you sign up for a new account on a site�it�s that your brain is super impressive for being able to.

On the other hand, multiplying big numbers or playing chess are new activities for biological creatures and we haven�t had any time to evolve a proficiency at them, so a computer doesn�t need to work too hard to beat us.

The idea of AGI is that we�d build a computer whose two major skills would be doing research on AI and coding changes into itself�allowing it to not only learn but to improve its own architecture. We�d teach computers to be computer scientists so they could bootstrap their own development. And that would be their main job�figuring out how to make themselves smarter.